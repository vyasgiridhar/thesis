\chapter{Design}\label{chapter:design}

The previous chapter has introduced the reader to the concept of matrix profiles and
their efficient computation and key concepts of development on Cerebras Wafer Scale Engine. The following chapter presents the top-level architecture as a combination
of a Driver Application and an Accelerated Kernel by utilizing this background.
The presented design constitutes a tiled version of the algorithm described in
Subsection 2.1.2 and can, therefore, decouple the problem size from the concrete kernel
implementation. Given the hardware capabilities of the WSE-2, a Tiled version of the Kernel is introduced in the chapter and explores the various types of tiling solutions that can be incorporated.\\

We then explore the various implications of resource allocation on the WSE-2 on the memory bandwidth, computation time and startup time.\\

We finally look at various scheduling approaches for large and small time series.


\section{Architecture}

The program is divided into 3 parts.
\subsection{CSL Kernel}

\begin{lstlisting}
    var row_iters: u8 = math_lib.min(n_x - exclusion_lower, n_y);

	for (@range(u8, row_iters)) | row | {
		var diag_max: u8 = math_lib.min(n_x - exclusion_upper + 1, n_x - row);

		var cur_dg_b: f32 = dg_b[row];
		var cur_df_b: f32 = df_b[row];
		var cur_norm_b: f32 = norm_b[row];
		for (@range(u8, exclusion_lower, diag_max, 1)) | diag | {
			var col: u8 = diag + row;
			// Calculate corr
			var corr: f32 = cov[diag] * norm_a[col] * cur_norm_b;
			if (corr > 0.99999946 and corr < 1.0) {
				corr = math_lib.ceil(corr);
			} else if (corr > 1.0) {
				corr = math_lib.floor(corr);
			}

			if (math_lib.isNaN(corr)) {
				// Smallest value possible is -1 so set to -2.
				corr = -2.0;
			}
			// Update cov
			var coeff: f32 = df_a[col] * cur_dg_b + dg_a[col] * cur_df_b;
			cov[diag] = cov[diag] + coeff;

			// Update profile
			update_profile(corr, P_a, P_i_a, row, col);
			update_profile(corr, P_b, P_i_b, col, row);
		}
	}
\end{lstlisting}

\subsection{Host Driver}
\subsection{Driver}

\section{Kernel Design}

The main SCAMP algorithm.

\begin{algorithm}
\caption{SCAMP Driver}\label{alg:SCAMP}
    \hspace*{\algorithmicindent} \textbf{Input} : Time Series $T$ of length \( n \in \mathbb{N} \) and subsequence length  \( m \in \mathbb{N} \) \\
    \hspace*{\algorithmicindent} \textbf{Output} : Matrix Profile $MP$ and Matrix Profile Index $MPI$
    \begin{algorithmic}[1]
        \State $df,dg,inv \gets PreComputeStatistics(T, m);$
        \State $QT_{init} \gets PreComputeInitialQTRow(T, m);$
        \State $rowAggregates, columnAggregates \gets (-\inf, -1);$
        \For{$iteration \gets 0$ \textbf{to} $\lceil \frac{n - m + 1}{1} - 1 \rceil$}
            \State $iteration_i \gets MatrixProfileKernel(QT_{init}, $df$, $dg$, $inv$);$
            \State $UpdateAggregates(rowAggregates, columnAggregates, iteration_i);$
        \EndFor
        \State $PostCompute(rowAggregates, columnAggregates);$\\
        \Return $MP, MPI;$
    \end{algorithmic}
\end{algorithm}
    
\section{Tiling}

\begin{itemize}
    \item SCAMP implements tiling.
    \item Talk about algorithm to compute tile positions.
\end{itemize}

\begin{algorithm}
    \caption{Tiled Kernel}\label{alg:Tiled}
        \hspace*{\algorithmicindent} \textbf{Input} : Time series $T_a$, $T_b$, $df$, $dg$, $inv$ and $args$ for current \textbf{Tile}.\\
        \hspace*{\algorithmicindent} \textbf{Output} : Row- and Column-Wise Aggregates for the current \textbf{Tile}
        \begin{algorithmic}[1]
            \For{$ProcessingElement \gets 0$ \textbf{to} $w * h$}
                \State $rowAggregates_{ProcessingElement} \gets (-\inf, -1);$
                \State $columnAggregates_{ProcessingElement} \gets (-\inf, -1);$
                \If{$args.full\_tile$ is $0$}
                    \Comment{Compute only top triangle}
                    \State $QT \gets computeQT(T_a, T_b);$
                    \For{$row \gets  0$ \textbf{to} $min(args.n_x - args.exclusion\_lower, args.n_y)$}
                        \For{$diag \gets args.exclusion\_lower$ \textbf{to} $min(args.n_x - args.exclusion\_upper + 1, args.n_x - row)$};
                            \State $col \gets row + diag;$
                            \State $correlation \gets QT_{diag} \cdot inv_{row} \cdot inv_{col};$
                            \If{$correlation$ $>$ $rowAggregate_{ProcessingElement, row}.value$}
                                \State $rowAggregate_{ProcessingElement, row}.value \gets (correlation, column);$
                            \EndIf
                            \If{$correlation$ $>$ $rowAggregate_{ProcessingElement, column}.value$}
                                \State $rowAggregate_{ProcessingElement, column}.value \gets (correlation, row);$
                            \EndIf
                        \EndFor
                    \EndFor
                \EndIf
            \EndFor
            \State $rowAggregates \gets (-\inf, -1);$
            \State $columnAggregates \gets (-\inf, -1);$
            \For{$ProcessingElement \gets 0$ \textbf{to} $w * h$}
                \State $Merge(rowAggregates, rowAggregates_{ProcessingElement});$
                \State $Merge(columnAggregates, columnAggregate_{ProcessingElement});$
            \EndFor
            \Return $rowAggregates, columnAggregates;$

        \end{algorithmic}
\end{algorithm}
    

\subsection{Trapezoidal}

\begin{itemize}
    \item Area = b * h
    \item Same as square.
    \item Need modification to create large number of extra tiles with no elements.
\end{itemize}

\subsection{Square}

\begin{itemize}
    \item Area = w x h
    \item Currently implemented in SCAMP
    \item Edge cases are rectangles.
    \item Diagonal tiles are upper triangles
\end{itemize}

\subsection{Triangles}

\begin{itemize}
    \item Area = 1/2 * b * h
    \item Currently implemented in SCAMP, can also be implemneted on the cerebras interface
    \item More jobs to be spread across limited number of PEs.
    \item More iterations on larger time series.
\end{itemize}


\section{Resource Allocation}
\subsection{Rectangle}


\subsection{Square}

\section{Scheduling}
\subsection{One Shot}
\subsection{Iterative}


