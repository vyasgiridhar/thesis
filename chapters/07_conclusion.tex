\chapter{Conclusion}\label{chapter:conclusion}


This work introduces an Matrix Profiling algorithm, based on the SCAMP algorithm, designed to compute matrix profiles on the Cerebras WSE-2. We provide an overview of concepts related to matrix profile computation and the architecture of the Cerebras WSE-2. Following this background, we delve into the specifics of our implementation, discussing the general architecture, including the division of computation between the Driver Application and Accelerated Kernel, as well as detailing two concrete kernel designs: the Vanilla and the Tiled Kernel.

We then address implementation considerations specific to the Cerebras WSE-2, outlining the steps involved in setting up and executing programs on the device. Subsequently, we conduct performance analyses using experiments performed on the Wafer Scale Cluster setup in the Edinburgh International Data Facility. These experiments include analyzing the performance capabilities of the Cerebras WSE-2 and deriving execution models for various tile sizes and memory consumption patterns. Notably, we highlight the challenge posed by the limited memory capacity of the processing elements (PEs) present in the device.

The insights derived from our experiments are valuable for understanding the potential of the Cerebras WSE-2 in practical applications. We also identify areas for future exploration, such as extending the implementation to handle row- and column-aggregates reduction on the wafer, optimizing inter-PE communication, implementing multi-stage \texttt{memcpy} to the device, and exploring strategies to utilize the entire wafer for calculation. Overall, we believe that the presented design can serve as a foundation for further research and development in this domain.